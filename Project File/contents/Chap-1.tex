\chapter{Project File}
\label{chp:intro}

\section{Original Instruction}
A feedback control system for a swinging robotic gymnast must be designed, implemented and verified. Feedback control loops must be designed that use the "legs" of the gymnast to swing the "body" of the gymnast from a "hanging" position to a "handstand" position and then balance the gymnast on top of the horizontal bar. A mathematical model for the dynamics of the swinging robotic gymnast system must be derived or sourced from literature. The dynamics must be analysed to propose an appropriate feedback control architecture that actuates the "legs" of the gymnast using feedback from sensors that measure the swinging motion of the gymnast on a horizontal bar. A practical demonstrator must be constructed and the correct operation must be demonstrated.

\includepdf[pages=1,pagecommand={\section{Project Proposal} \thispagestyle{empty} \label{sec:proposal}}, fitpaper=true]{./figs/proposal/Proposal.pdf}
\includepdf[pages=2-,pagecommand={\thispagestyle{empty}}, fitpaper=true]{./figs/proposal/Proposal.pdf}


\includepdf[pages=1,pagecommand={\section{Progress Report} \thispagestyle{empty} \label{sec:progress_report}}, fitpaper=true]{./figs/progress_report/progress_report.pdf}
\includepdf[pages=2-,pagecommand={\thispagestyle{empty}}, fitpaper=true]{./figs/progress_report/progress_report.pdf}



\includepdf[pages=1,pagecommand={\section{Preliminary Final} \thispagestyle{empty} \label{sec:preliminary_final}}, fitpaper=true]{./figs/draft/prelim_final.pdf}
\includepdf[pages=2-,pagecommand={\thispagestyle{empty}}, fitpaper=true]{./figs/draft/prelim_final.pdf}


\section{Weekly Progress Report}

\begin{comment}
\includepdf[pages=1,pagecommand={\section{Mechanical Drawings} \thispagestyle{empty} \label{sec:mech_drawings}}, fitpaper=true]{./figs/mech_drawings/mech_drawings.pdf}
\includepdf[pages=2-,pagecommand={\thispagestyle{empty}}, fitpaper=true]{./figs/mech_drawings/mech_drawings.pdf}

\includepdf[pages=1,pagecommand={\section{Microcontroller Settings} \thispagestyle{empty} \label{sec:micrcontroller_settings}}, fitpaper=true]{./figs/pinouts/acrobat_v4.pdf}
\includepdf[pages=2-,pagecommand={\thispagestyle{empty}}, fitpaper=true]{./figs/pinouts/acrobat_v4.pdf}
\end{comment}



\includepdf[pages=1,pagecommand={\section{PCB Front Copper Layout} \thispagestyle{empty} \label{sec:front_cu}}, fitpaper=true]{./figs/schematics/front_cu.pdf}

\section{Software Design}
\subsection{MATLAB Code}
\subsection{title}

\section{Tikz Code}
\subsection{AND Gate Circuit}

\begin{verbatim}
		\begin{tikzpicture}[every path/.style={},>=triangle 45,circuit logic US, every circuit symbol/.style={}]
	% Logic Gates
	\node[and gate,inputs={nn}, point right] (and1) at (2,-1)    {};
	\node[and gate,inputs={nn}, point right] (and2) at (2,-2)    {};
	\node[not gate, point right] (not1) at (0,-0.5) {};
	
	
	\draw (not1.output)[thick] -| (1,-0.9) -- (and1.input 1);
	
	%Outputs
	\draw (and1.output) [thick] -- (3,-1) node[ocirc,label={right:IN 1} ](in1) {};
	\draw (and2.output) [thick]-- (3,-2) node[ocirc,label={right:IN 2} ](in2) {};
	
	
	%inputs DIR
	\draw ([xshift=-1.7cm]not1.input) [thick] -| ([xshift=-0.5cm]not1.input) -- (not1.input) {};
	\draw ([xshift=-0.5cm]not1.input)[thick] |- (and2.input 1) {};
	
	% input PWM
	\draw ([xshift=-3.6cm]and2.input 2) [thick] -- (and2.input 2){};
	\draw ([xshift=-0.65cm]and2.input 2) [thick] |- (and1.input 2){};
	
	\draw ([xshift=-1.7cm]not1.input) node[ocirc,label={left:DIR} ](dir) {};
	\draw ([xshift=-3.6cm]and2.input 2) node[ocirc,label={left:PWM} ](pwm) {};
	
	% dots
	\draw [*-]([xshift=-0.5cm,yshift=0.08cm]not1.input){};
	\draw [*-]([xshift=-0.65cm,yshift=0.08cm]and2.input 2){};
	\end{tikzpicture}
\end{verbatim}


\subsection{AND Gate Circuit Waveform}
\begin{verbatim}
	\begin{tikztimingtable}
		PWM Signal 10\,kHz	   & H 12{2C} G\\
		Direction Signal  	   & L  12L  12H \\
		A						   & L 12L 12H \\
		IN1				   	   & H 6{2C} 12{L}  \\
		IN2			   		   & L 7{2L} 5{2C}  \\
		%Coarse Pulse             & 3L 16H 6L \\
		%Coarse Pulse - Delayed 1 & 4L N(B2) 16H N(B6) 5L \\
		%Coarse Pulse - Delayed 2 & 5L N(B3) 16H N(B7) 4L \\
		%Coarse Pulse - Delayed 3 & 6L 16H 3L \\
		\\
		%Final Pulse Set          & 3L 16H N(B5) 6L \\
		%Final Pulse $\overline{\mbox{Reset}}$ & 6L N(B4) 16H 3L \\
		%Final Pulse              & 3L N(B1) 19H N(B8) 3L \\
		\extracode
		\tablerules
		%0\begin{pgfonlayer}{background}
		%   \foreach \n in {1,...,1}
		%    \draw [help lines] (A\n) -- (B\n);
		%\end{pgfonlayer}
	\end{tikztimingtable}
\end{verbatim}

\subsection{Electronic System Overview}
\begin{verbatim}
	\pgfdeclarelayer{background}
	\pgfdeclarelayer{foreground}
	\pgfsetlayers{background,main,foreground}
	
	% Define block styles  
	\tikzstyle{block}=[draw, fill=blue!20, text width=7.0em, text centered,
	minimum height=1.5em,drop shadow]
	\tikzstyle{blocks} = [block, rounded corners, drop shadow]
	\tikzstyle{texto} = [above, text width=6em, text centered]
	\tikzstyle{linepart} = [draw, thick, color=black!50, -latex', dashed]
	\tikzstyle{line} = [draw, thick, color=black!50, -latex']
	\tikzstyle{ur}=[draw, text centered, minimum height=0.01em]
	
	% Define distances for bordering
	\newcommand{\blockdist}{1.3}
	\newcommand{\edgedist}{1.5}
	
	\newcommand{\external}[2]{node (e#1) [blocks]
		{External 12V Supply\\{\scriptsize\textit{#2}}}}
	
	\newcommand{\regulator}[2]{node (r#1) [blocks]
		{Voltage Regulation\\{\scriptsize\textit{#2}}}}
	
	\newcommand{\uC}[2]{node (uC#1) [blocks]
		{$\mu$Controller\\{\scriptsize\textit{#2}}}}
	
	\newcommand{\uart}[2]{node (uart#1) [blocks]
		{PC UART Interface\\{\scriptsize\textit{#2}}}}
	
	\newcommand{\prog}[2]{node (prog#1) [blocks]
		{Programming / Debug Interface\\{\scriptsize\textit{#2}}}}
	
	\newcommand{\motor}[2]{node (motor#1) [blocks]
		{Motor\\{\scriptsize\textit{#2}}}}
	
	\newcommand{\sigcond}[2]{node (sigcond#1) [blocks]
		{Signal Conditioning\\{\scriptsize\textit{#2}}}}
	
	\newcommand{\encdig}[2]{node (encdig#1) [blocks]
		{Digital Logic Circuit\\{\scriptsize\textit{#2}}}}
	
	\newcommand{\pc}[2]{node (pc#1) [blocks]
		{PC\\{\scriptsize\textit{#2}}}}
	
	\newcommand{\physical}[2]{node (physical#1) [blocks]
		{Physical Model\\{\scriptsize\textit{#2}}}}
	
	\newcommand{\motordriver}[2]{node (motordriver#1) [blocks]
		{Motor Driver\\{\scriptsize\textit{#2}}}}
	
	\newcommand{\digitlogic}[2]{node (digitlogic#1) [blocks]
		{Digital Logic Circuit\\{\scriptsize\textit{#2}}}}
	
	\newcommand{\encoder}[2]{node (encoder#1) [blocks]
		{Hall Effect Encoder\\{\scriptsize\textit{#2}}}}
	% Draw background
	
	
	\newcommand{\transreceptor}[3]{%
		\path [linepart] (#1.east) -- node [above]
		{\scriptsize Transreceptor #2} (#3);}
	
	\begin{document}
		\begin{tikzpicture}[scale=0.7,transform shape]
		
		% Draw diagram elements
		\path \external {1}{DC Power Supply};
		\path (e1.east)+(2.0,0.0) \physical{1}{Potentiometer};
		\path (e1.south)+(0.0,-1.5) \regulator{1}{5V, 3.3V};
		\path (r1.south)+(0.0,-1.5) \uC{1}{ARM M0 STM32F030C6};
		
		% PC 
		\path (e1.west)+(-2.5,0) \pc{1}{};
		
		% PC UART Interface
		\path (r1.west)+(-6,0) \uart{1}{FT230XS};
		
		%Programming/Debug Interface
		\path (r1)+(-4.05,0) \prog{1}{Serial Wire Debug};
		
		%Signal Conditioning
		\path (uC1.east)+(2.0,0) \sigcond{1}{MCP602 OpAmp};
		
		%JK Flipflops
		\path (uC1.west)+(-2.0,-3.0) \encdig{1}{J-K Flipflop \& Nor's};
		
		% Motor
		\path (uC1.south) + (0,-5) \motor{1}{DC Brushed Motor};
		
		% Digital Logic: Logic Level Convertes
		\path (uC1.south) + (0,-2) \digitlogic{1}{Logic Level Converters \&  Direction Control};
		
		% Motor Driver
		\path (digitlogic1.east)+(2.0,0) \motordriver{1}{MC33887};
		
		%Hall Effect Enconder
		\path (encdig1.south)+ (0,-1.8) \encoder{1}{Mounted On Motor};
		
		
		
		
		% Draw arrows between elements
		\path [line] (e1.south) -- node [above] {} (r1);
		\path [line] (r1.south) -- node [above] {} (uC1);
		
		% uC to UART
		\path [line] (uC1.west) -| node [below] {} (uart1);
		
		% uC to Programming/Debug Interface
		\path [line] (uC1.west)+(0,0.2) -| node [below]{}(prog1); 
		
		% JK FlipFlops
		%\path [line] (uC1.west)+(0,-0.2) -| node [above]{}(encdig1); 
		
		\draw[->] (encdig1) |- ([yshift=-0.2cm]uC1.west);
		
		
		\path [line] (sigcond1.west) -- node[right]{}(uC1);
		
		\path [line] (physical1.south) -- node[above]{}(sigcond1);
		
		% Motor Driver to signal Conditioning
		\path [line] (motordriver1.north) -- node[below]{}(sigcond1);
		
		% PC UART Interface -> PC
		\path [line] (uart1.north) |- node[left]{}(pc1);
		
		% Programming/Debug Interfac -> PC
		\path [line] (prog1.north) -- node[below]{}(pc1);
		
		% Motor Driver -> Motor
		\path [line] (motordriver1.south) |- node[right]{}(motor1); 
		
		% Microcontroller -> Digitical logic
		\path [line] (uC1.south) -- node[above]{}(digitlogic1);
		
		\path [line] (digitlogic1.east) -- node[left]{}(motordriver1);
		
		
		\path [line] (encoder1.north) -- node[below]{}(encdig1);
		
		
		\path [line] (motor1.west) -- node[right]{}(encoder1);
		
		\begin{pgfonlayer}{background}
		\path (uart1.west -| physical1.east) node (a) {};
		\path (motor1.south -| physical1.south)+(+0.5,-0.3) node (b) {};
		\path (digitlogic1.south |- motor1.east)+(+0.5,0.5) node (c) {};
		
		\path[fill=yellow!20,rounded corners, draw=black!50, dashed]
		([xshift=-0.5cm,yshift=1cm]uart1.west) rectangle ([xshift=0.5cm,yshift=-2cm]motordriver1.east);           
		\path (digitlogic1.north west)+(-0.2,0.2) node (a) {};
		
		\end{pgfonlayer}
		
		\path ([xshift=-4.5cm,yshift=-0.5cm]encdig1.south) node (meep) {PCB Boundary};
		
		%\path (wa.south)+(0,-\blockdist/5) node (meep) {System Boundary};
		
		
		\end{tikzpicture}
\end{verbatim}

\subsection{Motor Driver Circuit}
\begin{verbatim}
\end{verbatim}

\subsection{Freebody Diagram}
\begin{verbatim}
\end{verbatim}

\subsection{Inertia Diagram}
\begin{verbatim}
\end{verbatim}

\subsection{JK XOR Circuit}
\begin{verbatim}
\end{verbatim}

