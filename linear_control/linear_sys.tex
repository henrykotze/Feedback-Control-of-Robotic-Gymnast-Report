\documentclass{article}

\usepackage{tikz}
\usetikzlibrary{shapes,arrows}
\begin{document}


\tikzstyle{block} = [draw, fill=blue!20, rectangle, 
    minimum height=3em, minimum width=6em]
\tikzstyle{sum} = [draw, fill=blue!20, circle, node distance=1cm]
\tikzstyle{input} = [coordinate]
\tikzstyle{output} = [coordinate]
\tikzstyle{pinstyle} = [pin edge={to-,thin,black}]

% The block diagram code is probably more verbose than necessary
\begin{tikzpicture}[auto, node distance=2cm,>=latex']
    % We start by placing the blocks
    
    \node [block, name=A](A){$\boldmath{A}$} {};
    \node [block, below of=A] (K) {$-\boldmath{K}$};
    \node [output, right of=A] (output) {};

    % Once the nodes are placed, connecting them is easy. 
   	\draw [->] (A) -- node [name=y] {}(output);
   	\draw [->] (y) |- (K);
   	%\draw [->] (K) -- (A);
   	
    %\draw [->] (output) |- node[] {} (K);
    %\draw [->] (system) -- node [name=y] {$y$}(output);
    %\draw [->] (y) |- (measurements);
    %\draw [->] (measurements) -| node[pos=0.99] {$-$} 
     %   node [near end] {$y_m$} (sum);
\end{tikzpicture}

\end{document}