\chapter{Introduction}
\label{chp:intro}


%%%%%%%%%%%%%%%%%%%%%%%%%%%%%%%%%%%%%%%%%%%%%%%%%%%%%%%%%%%%%%%%%%%%%%%
\section{Problem Statement}

A feedback control system for a swinging robotic gymnast must be designed, implemented and verified. Feedback control loops must be designed that use the "legs" of the gymnast to swing the "body" of the gymnast from a "hanging" position to a "handstand" position and then balance the gymnast on top of the horizontal bar. A mathematical model for the dynamics of the swinging robotic gymnast system must be derived or sourced from literature. The dynamics must be analysed to propose an appropriate feedback control architecture that actuates the "legs" of the gymnast using feedback from sensors that measure the swinging motion of the gymnast on a horizontal bar. A practical demonstrator must be constructed and the correct operation must be demonstrated.



The feedback control of a robotic gymnast entails to model the behavior the of a gymnast hanging from a horizontal bar and tries to swing himself upwards using his legs until he balance himself in the inverted position. The problem is modelled as a double pendulum: where one end is connected to a fixed hinge, the 2 pendulums are connected with a hinge between eachother and there is some torque actuating the lower pendulum.


\section{Literature Study}
	% WHAT you are going to present in this chapter/section
% WHY you are presenting it, and
% HOW you are going to present it	
The fundamental concepts that provides the foundation for the concepts discussed in the report will be summarise here. It acts as a refresher for those familiar to control theory. 

The ordinary differential equations (ODE's) describing a system can be arranged as a set of linear differential equations. Describing a system in such a way is known as the State Space design approach, where the solution is the trajectory of the chosen state variables.\cite{textbook}

These ODE's are required to be written as vectors in the state-variable form seen in equation (\ref{eq:statespace1}) and (\ref{eq:statespace2})
\begin{equation} \label{eq:statespace1}
\centering
\boldsymbol{\dot{x}} = \boldsymbol{A}\boldsymbol{x} + \boldsymbol{B}u
\end{equation}
\begin{equation} \label{eq:statespace2}
\centering
\boldsymbol{y} = \boldsymbol{C}\boldsymbol{x} + Du
\end{equation}
The \textit{n}th-column vector $\boldsymbol{x}$ is called the state of the system for a \textit{n}th-order system. The \textbf{A} matrix is the system matrix, containing \textit{n}$\times$\textit{n} elements and the input matrix is the \textit{n}$\times 1$ \textbf{B} matrix. \textbf{C} is a $1\times$\textit{n} row matrix called the output matrix and the scalar D is known as the direct transmission term \cite{textbook}.

A system parameter of great interest to control engineers are the poles of the system. It provides the characteristic response of the system starting at a initial condition with no forcing function. These poles,\textbf{s}, are the natural frequencies of the system and the state space representation allow these poles to be easy identified. The poles are the solution to the eigenvalue problem of the \textbf{A} matrix shown in equation (\ref{eq:statespace_eigen}) \cite{textbook}.
\begin{equation} \label{eq:statespace_eigen}
\centering
\text{det}(s\boldsymbol{I} - \boldsymbol{A}) = 0
\end{equation}

The poles of the system can be assigned new positions to satisfy dynamic response specification by introducing feedback. The feedback is a linear combination of the state variables $\boldsymbol{x}$ resulting in the input of the system $u$ to be transformed as seen in equation (\ref{eq:feedbackgain}) and represented in Figure \ref{fig:linearSys}. Substituting equation (\ref{eq:feedbackgain}) into (\ref{eq:statespace1}) the characteristic equation is now describe as (\ref{eq:closedSysFeedback}). The corresponding characteristic equation is: $$\alpha_{s}=(s-s_{1})(s-s_{2})\ldots(s-s_{n}) $$ This shows by selecting the correct gain matrix \textbf{K} the poles of the system can be moved to a desired position.
\begin{equation} \label{eq:feedbackgain}
\centering
u = -\boldsymbol{K}\boldsymbol{x}
\end{equation}

\begin{figure}
	\centering
	% System Combination
% Harish K Krishnamurthy <www.ece.neu.edu/~hkashyap/>
\documentclass{article}

\usepackage{tikz}
\usetikzlibrary{shapes,arrows,shadows}
\usepackage{amsmath,bm,times}
\newcommand{\mx}[1]{\mathbf{\bm{#1}}} % Matrix command
\newcommand{\vc}[1]{\mathbf{\bm{#1}}} % Vector command

\begin{document}
	% Define the layers to draw the diagram
	\pgfdeclarelayer{background}
	\pgfdeclarelayer{foreground}
	\pgfsetlayers{background,main,foreground}
	
	% Define block styles used later
	
	\tikzstyle{sensor}=[draw, fill=blue!20, text width=5em, 
	text centered, minimum height=2.5em,drop shadow]
	\tikzstyle{ann} = [above, text width=5em, text centered]
	\tikzstyle{wa} = [sensor, text width=10em, fill=red!20, 
	minimum height=6em, rounded corners, drop shadow]
	\tikzstyle{sc} = [sensor, text width=13em, fill=red!20, 
	minimum height=10em, rounded corners, drop shadow]
	
	% Define distances for bordering
	\def\blockdist{2.3}
	\def\edgedist{2.5}
	
	\begin{tikzpicture}
	\node (wa) [sensor]  {$\boldsymbol{\dot{x}}= \boldsymbol{A}\boldsymbol{x}+\boldsymbol{B}$};
	\path (wa.south)+(0,-1) node (feedback) [sensor] {$u = -\boldsymbol{K}\boldsymbol{x}$};
	
	\path (wa.east)+(\blockdist/1.5,0) node (C) [sensor] {$\boldsymbol{C}$};
	\path (C.east)+(\blockdist/1.5,0) node (Y) [sensor] {$\boldsymbol{y}$};
	
	
	\path [draw, ->,thick] (wa.east) -- node [above] {} 
	(C.west);
	
	\path [draw, ->,thick] (C.south) |- node [above] {} 
	(feedback.east);
	
	\path [draw, ->,thick] (C.east) -- (Y.west);
	
	\path [draw, ->,thick] (feedback.west) -| ([xshift=-1cm]wa.west) -- (wa.west) {};
	
	%\path [draw, ->,] (C.east) -- node [above] {} 
	%	(Y.west);
	
	%\path (wa.south) +(0,-\blockdist) node (asrs) {System Combination - Training};
	
	%\begin{pgfonlayer}{background}
	%   \path (asr1.west |- asr1.north)+(-0.5,0.3) node (a) {};
	%  \path (wa.south -| wa.east)+(+0.5,-0.3) node (b) {};
	% \path (C.east |- asrs.east)+(+0.5,-0.5) node (c) {};
	
	%\path[fill=yellow!20,rounded corners, draw=black!50, dashed]
	%   (a) rectangle (c);           
	% \path (asr1.north west)+(-0.2,0.2) node (a) {};
	
	%\end{pgfonlayer}
	
	\end{tikzpicture}
	
\end{document}}
	\caption{State Space Representation with Feedback Gain}
	\label{fig:linearSys}
\end{figure}

\begin{equation} \label{eq:closedSysFeedback}
\centering
\text{det}[s\boldsymbol{I}-(\boldsymbol{A}-\boldsymbol{B}\boldsymbol{K})] = 0
\end{equation}

The classical approach to controlling a system is by implementing a controller which reacts on the error of the desired state and the current state. These controllers are more commonly known as PID-controllers where the controller equation is shown in (\ref{eq:PID}).

\begin{equation} \label{eq:PID}
\centering
u(t) = K[ e(t)+K_{I}\int_{0}^{t}e(\tau)d\tau +K_{D}\frac{de(t)}{dt}]
\end{equation}

Each term represent an effect it has on the system response when the PID-controller is implemented shown in Figure \ref{fig:PIDcontroller}. If the system or plant is assumed to be a second-order differential equation represented by:
\begin{equation} \label{eq:PID_system}
\centering
\dddot{q}+(2\zeta\omega_{n}+K_{D})\ddot{q}+(\omega_{n}^2+K_{P})\dot{q}+K_{I} = 0
\end{equation}

From equation (7) it is visible that by tuning the PID constants the response of the system can controlled.

\section{System Overview}
%WHAT you are going to present in this chapter/section
%WHY you are presenting it, and
%HOW you are going to present it
\begin{figure}[h]
	\centering
	\newcommand{\mx}[1]{\mathbf{\bm{#1}}} % Matrix command
\newcommand{\vc}[1]{\mathbf{\bm{#1}}} % Vector command


% Define the layers to draw the diagram
\pgfdeclarelayer{background}
\pgfdeclarelayer{foreground}
\pgfsetlayers{background,main,foreground}

% Define block styles used later

\tikzstyle{sensor}=[draw, fill=red!20, text width=5em, 
text centered, minimum height=2.5em,drop shadow,rounded corners]
\tikzstyle{ann} = [above, text width=5em, text centered]
\tikzstyle{wa} = [sensor, text width=10em, fill=red!20, 
minimum height=6em, rounded corners, drop shadow]
\tikzstyle{sc} = [sensor, text width=13em, fill=red!20, 
minimum height=10em, rounded corners, drop shadow]

% Define distances for bordering
\def\blockdist{2.3}
\def\edgedist{2.5}

\begin{tikzpicture}[scale=1.2]
\centering
\node (wa) [wa]  {\textbf{Electronic Design} \\ PCB \\ Signal Conditioning};
\path (wa.west)+(-\blockdist,0) node (asr1) [wa] {\textbf{External Computer} \\ Controller \\ Data Aquisition };
\path (wa.east)+(\blockdist,0) node (vote) [wa] {\textbf{Mechanical Design} \\ State variables \\ };
\path (wa.north)+(0,\blockdist/2) node (pow) [sensor] {\textbf{External Power}};
\path (asr1.north)+(0,\blockdist/2) node (human) [sensor] {\textbf{Human Input} \\ };


\path [draw, <->,thick] (asr1.east) -- node [above] {} 
(wa.west) ;

\path [draw, <->,thick] (wa.east) -- node [above] {} 
(vote.west);

\path [draw, <->,thick] (human.south) -- node [above] {} 
(asr1.north);   

\path [draw,thick, ->] (pow.south) -- node [above] {} 
(wa.north);   


\path (wa.south) +(0,-\blockdist/2) node (asrs) {};
 \path (wa.south)+(0,-\blockdist/5) node (meep) {System Boundary};


\begin{pgfonlayer}{background}
\path (asr1.west |- asr1.north)+(-0.5,0.3) node (a) {};
\path (wa.south -| wa.east)+(+0.5,-0.3) node (b) {};
\path (vote.east |- asrs.east)+(+0.5,0.5) node (c) {};

\path[fill=yellow!20,rounded corners, draw=black!50, dashed]
(a) rectangle (c);           
\path (asr1.north west)+(-0.2,0.2) node (a) {};

\end{pgfonlayer}

\end{tikzpicture}
	\caption{System Overview of the Feedback Control of Robotic Gymnast}
	\label{fig:system_overview}
\end{figure}


Figure \ref{fig:system_overview} provides an overview of the various subsystems the project will contain. The project is subdivided into these subsystems being developed separately with little interaction between the each other. An brief overview on each subsystem will be presented here.\\

The external computer communicates with the electronic design sending instruction to start the system and for debugging purposes. The external computer will receive data from the electronic design and allows the verification of system parameters.

The electronic design acts as the middle-man between the mechanical design and the external computer. It provides instructions to the mechanical design components based on the controller while sending data to the external computer. The electronic design contains the Printed Circuit Board (PCB) that conditions all the signals for processing.\\

The mechanical design is responsible for creating a physical model that represents the mathematical model describing the system. The correct sensors must be selected to measure the state variables and providing interfaces for the electronic design.

\section{Project Execution}
%WHAT you are going to present in this chapter/section
%WHY you are presenting it, and
%HOW you are going to present it
% write as if i already happened

The execution of the project occurs in a sequence of steps. First the mathematical model of the system is derived by using the appropriate approaches. The derived mathematical model is then implemented on a simulation program where the dynamics of the system can be verified and inspected.\\

Following the successful implementation of the mathematical mode the various controllers will be designed and implemented on the simulation program. The behaviour of the simulated system will be inspected.\\

From the simulation results the mechanical design specification will be determined. The mechanical design will then commence to create a physical model that provides an acceptable representation of the mathematical model.\\

During manufacturing of the mechanical design the electronic design will start. Conceptual designs will be done to determine the various designs capable of meeting the requirements. The electronic design will be tested to ensure it performs as designed with the opportunity to create revisions of the design.\\

The programming of the microcontroller, from now on as $\mu$C, will start. This includes the programming of the controller, data acquisition and the conversions of the sampled data.\\

System identification will then occur to determine the various parameters required by the controllers. These newly determined parameters will then be implemented on the simulation program. New controllers will be designed and verified in simulation.

The new controllers are then implemented onto the $\mu$C for the system experiments to start. From these system experiments the responses of the experiment will be compared by those of the simulation.\\

The report was written throughout the sequence of steps mentioned above and will be completed and reviewed at the end.


\section{Report Outline}
%WHAT you are going to present in this chapter/section
%WHY you are presenting it, and
%HOW you are going to present it

A brief overview of each chapter in the report will provided here. It acts as primer for the reader and the easy identification of section that may interest the reader more.\\

Chapter 2 explain the system concepts that is reference throughout the report. It contains the mathematical derivation of the robotic gymnast and the simulated model. The system parameters with system identification test are shown and whether the simulated models is a acceptable representation of the physical model.

Chapter 3 describes the design paradigm to swinging and 
 
