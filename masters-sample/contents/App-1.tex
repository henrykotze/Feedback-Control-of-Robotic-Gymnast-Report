\chapter{Derivation of the Double Pendulum}
\label{chp2:concept_model}

\section{Derivation of the Mathematical Model}
\label{sec:math_model}

$$x_{1}= l_{1}\cos(\theta)$$
$$y_{1} = -l_{1}\sin(\theta)$$

$$x_{2} = L_{1}\sin(\theta) + l_{2}\sin(\theta + \phi)$$
$$y_{2} = -L_{1}\cos(\theta) - l_{2}\cos(\theta + \phi)$$

$$\dot{x_{2}} = L_{1}\cos(\theta)\dot{\theta} - l_{2}\cos(\theta+\phi)(\dot{\theta}+\dot{\phi}) $$
$$\dot{y_{2}} = L_{1}\sin(\theta)\dot{\theta}+l_{2}\sin(\theta+\phi)(\dot{\theta}+\dot{\phi})$$

$$x_{2}^2 = L_{1}^2\cos(\theta)^2\theta^2 +l_{2}^2\cos(\theta+\phi)^2(\dot{\theta}+\dot{\phi})^2 + 2L_{1}l_{2}\dot{\theta}(\dot{\theta}+\dot{\theta})\cos(\theta)\cos(\theta+\phi)$$
$$y_{2}^2 = L_{1}^2\sin(\theta)^2\theta^2 +l_{2}^2\sin(\theta+\phi)^2(\dot{\theta}+\dot{\phi})^2 + 2L_{1}l_{2}\dot{\theta}(\dot{\theta}+\dot{\theta})\sin(\theta)\sin(\theta+\phi)$$

$$x_{2}^2+y_{2}^2 = L_{1}^2\theta^2[\cos(\theta)^2+\sin(\theta)^2]+l_{2}^2(\dot{\theta}+\dot{\phi})^2[\cos(\theta+\phi)^2+\sin(\theta+\phi)^2] +$$
$$ 2L_{1}l_{2}\dot{\theta}(\dot{\theta}+\dot{\phi})[\cos(\theta)\cos(\theta+\phi)+\sin(\theta)\sin(\theta+\phi)]$$	

Using the following trigonometric identities $$ \cos(\gamma)^2 + \sin(\gamma)^2 = 1 $$ 
$$ \cos(\gamma)\cos(\alpha)+\sin(\gamma)\sin(\alpha) = \cos(\gamma - \alpha) $$ the above equation resolves to: $$ V_{2}^2 = L_{1}\dot{\theta}^2+l_{2}^2(\dot{\theta}+\dot{\phi})^2 + 
2L_{1}l_{2}(\dot{\theta}+\dot{\phi})\dot{\theta}\cos(\phi)$$

The kinetic energy in the system consist of the fixed rotation of the underactuated  pendulum and the rotation and velocity of the actuated pendulum.

$$ T = \frac{1}{2}I_{A}\dot{\theta}^2 + \frac{1}{2}I_{B}(\dot{\theta}+\dot{\phi})^2 + \frac{1}{2}m_{2}V_{2}^2$$
$$ T = \frac{1}{2}I_{A}\dot{\theta}^2 + \frac{1}{2}I_{B}(\dot{\theta}+\dot{\phi})^2 + \frac{1}{2}m_{2}[L_{1}\dot{\theta}^2+l_{2}^2(\dot{\theta}+\dot{\phi})^2 + 
2L_{1}l_{2}(\dot{\theta}+\dot{\phi})\dot{\theta}\cos(\phi)]^2$$

The potential energy in the system is defined as
$$V=-m_{1}gl_{1}\cos(\theta)-m_{2}g[L_{1}\cos(\theta)+l_{2}\cos(\theta+\phi)]$$

The Lagrange is defined as 
$$\mathcal{L}=T-V$$
$$\mathcal{L} = \frac{1}{2}I_{A}\dot{\theta}^2 + \frac{1}{2}I_{B}(\dot{\theta}+\dot{\phi})^2 + \frac{1}{2}m_{2}[L_{1}\dot{\theta}^2+l_{2}^2(\dot{\theta}+\dot{\phi})^2 + 
2L_{1}l_{2}(\dot{\theta}+\dot{\phi})\dot{\theta}\cos(\phi)]^2+m_{1}gl_{1}\cos(\theta)+$$
$$m_{2}g[L_{1}\cos(\theta)+l_{2}\cos(\theta+\phi)]$$

$$\frac{\partial\mathcal{L}}{\partial\theta} = -m_{1}gl_{1}\sin(\theta)-m_{2}gL_{2}\sin(\theta)-m_{2}gl_{2}\sin(\theta+\phi)$$
$$\frac{d}{dt}\frac{\partial\mathcal{L}}{\partial\dot{\theta}} = I_{A}\ddot{\theta}+I_{B}\ddot{\theta}+I_{B}\ddot{\phi}+m_{2}L_{1}^2\ddot{\theta}+m_{2}l_{2}^2\ddot{\theta}+m_{2}l_{2}\ddot{\phi}+2m_{2}L_{1}l{2}\ddot{\theta}\cos(\phi)-2m_{2}L_{1}l_{2}\dot{\theta}\dot{\phi}\sin(\phi)+$$
$$m_{2}L_{1}l_{2}\ddot{\phi}\cos(\phi)-m_{2}L_{1}l_{2}\dot{\phi}^2\sin(\phi)$$


$$\frac{\partial\mathcal{L}}{\partial\phi} = -m_{2}L_{1}l_{2}(\dot{\theta}+\dot{\phi})\dot{\theta}\sin(\phi)-m_{2}gl_{2}\sin(\theta+\phi)$$

$$\frac{d}{dt}\frac{\partial\mathcal{L}}{\partial\dot{\phi}}=I_{B}\ddot{\theta}+I_{B}\ddot{\phi}+m_{2}l_{2}^2\ddot{\theta}+m_{2}l_{2}^2\ddot{\phi}+m_{2}L_{1}l_{2}\ddot{\theta}\cos(\phi)-m_{2}L_{1}l_{2}\dot{\theta}\dot{\phi}\sin(\phi)$$

The differential equation describing the dynamics of the system is
$$\frac{d}{dt}\frac{\partial\mathcal{L}}{\partial\vec{\dot{q}}}-\frac{\partial\mathcal{L}}{\partial q} = B(\dot{q})+\tau(q)$$ 
where  $ q = 
\begin{bmatrix}
\theta \\
\phi
\end{bmatrix}
$

\subsection{Ball mass and inertia parameters}

Consider a volume element $\mathrm{d}V$ with respect to a static base $S$ of
an arbitrary solid body with  density $\rho$. The mass of the body is
obtained by integrating over the volume of the body,
\begin{equation}
    m = \int\limits_{\mathrm{body}} \rho\, \mathrm{d}V
    \label{eq:BMass-dif}
\end{equation}

In figure~\ref{fig:BallDef}, a ball with radius $R_{i}$ and uniform density
$\rho_i$ is depicted. The mass of the ball is after integration of
equation~\eqref{eq:BMass-dif}
\begin{equation}
    m_i = \tfrac{4}{3} \pi \rho_i\, R_i^3 .
    \label{eq:BMass}
\end{equation}


%----------------------------------------------------------------------------
\endinput
