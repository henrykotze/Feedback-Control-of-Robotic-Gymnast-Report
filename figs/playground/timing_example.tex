\documentclass{article}
\usepackage{tikz-timing}[2009/05/15]
\pagestyle{empty}
\def\degr{${}^\circ$}

\newif\ifcomment

% Activate the following line to compile document with comments:
\commenttrue

\begin{document}

\ifcomment
\newcommand{\qv}[1]{`\texttt{#1}'}

\title{TikZ-Timing Example}
\author{Martin Scharrer\\martin@scharrer-online.de}
\maketitle

\section{Step-By-Step Guide}
\subsection{Intro}
This example shows how to make a timing diagram with the \texttt{tikz-timing} 
package. This timing diagram was used by the package author in a recent work and 
shows several clock and pulse signals. The relationship between the clock and 
signal edges is shown using horizontal lines.

Timing diagrams like this can be done using the \texttt{tikztimingtable} 
environment which has the same syntax as a \texttt{tabular} environment with two 
columns. The first column holds the signal name, the second one the timing 
characters. See the package manual for detailed information about them.

\subsection{Clock Signals}
Clock signals can be generated with the \qv{C} character.  Here we want a clock 
signal which is two unit width wide and without slope, so we use \qv{2C} for 
every half clock period.  We also use the \emph{repetition mode} to get the 
number of clock periods we want: \qv{12\{2C\}}.

To shift the different clocks horizontal we use single \qv{H} or \qv{L} 
characters at the start. Also the \emph{initial character} is used so that the 
clock signals start with an edge if needed. The \emph{glitch} character \qv{G} 
is used at the end of two clocks so that the signal ends with an edge.

\par\bigskip
\begin{verbatim}
\begin{tikztimingtable}
  Clock 128\,MHz 0\degr    & H   12{2C} G \\ % ends with edge
  Clock 128\,MHz 90\degr   & [C] 12{2C} C \\ % starts with edge
  Clock 128\,MHz 180\degr  & C   12{2C} G \\ % ends with edge
  Clock 128\,MHz 270\degr  &     12{2C} C \\
\end{tikztimingtable}
\end{verbatim}
\begin{tikztimingtable}
  Clock 128\,MHz 0\degr    & H   12{2C} G \\ % ends with edge
  Clock 128\,MHz 90\degr   & [C] 12{2C} C \\ % starts with edge
  Clock 128\,MHz 180\degr  & C   12{2C} G \\ % ends with edge
  Clock 128\,MHz 270\degr  &     12{2C} C \\
\end{tikztimingtable}

\subsection{Pulse and other Signals}
The pulses are simply set using \qv{H} (high) or \qv{L} (low) signals. The 
number before the characters is the width. Note that \qv{LLL}, \qv{3L}, \qv{2LL} 
or \qv{3\{L\}} all result in the same signal: a low signal with three times the 
unit width.

An empty line can be inserted to produce a vertical space between groups of 
signals.

\begin{verbatim}
\begin{tikztimingtable}
  Coarse Pulse                          & 3L 16H 6L \\
  Coarse Pulse - Delayed 1              & 4L 16H 5L \\
  Coarse Pulse - Delayed 2              & 5L 16H 4L \\
  Coarse Pulse - Delayed 3              & 6L 16H 3L \\
  \\ % Gives vertical space
  Final Pulse Set                       & 3L 16H 6L \\
  Final Pulse $\overline{\mbox{Reset}}$ & 6L 16H 3L \\
  Final Pulse                           & 3L 19H 3L \\
\end{tikztimingtable}
\end{verbatim}
\begin{tikztimingtable}
  Coarse Pulse                          & 3L 16H 6L \\
  Coarse Pulse - Delayed 1              & 4L 16H 5L \\
  Coarse Pulse - Delayed 2              & 5L 16H 4L \\
  Coarse Pulse - Delayed 3              & 6L 16H 3L \\
  \\ % Gives vertical space
  Final Pulse Set                       & 3L 16H 6L \\
  Final Pulse $\overline{\mbox{Reset}}$ & 6L 16H 3L \\
  Final Pulse                           & 3L 19H 3L \\
\end{tikztimingtable}

\subsection{Drawing Relationship Lines}
To draw own lines inside the timing diagram use \verb+\extracode+ after the last 
table line. Any kind of PGF/TikZ macros can be placed between this macro and the 
end of the table environment. The coordinate origin is on the baseline of the 
first timing line. See also the package manual for more information.

To simplify the drawing process we simply add named notes into the timing line 
using the \qv{N(\textit{name})} character. All start points are called 
\qv{A\textit{number}} and all corresponding end points are called 
\qv{B\textit{number}}. The notes are placed at the appropriate positions and do 
not produce any graphic but only name the current position. Some repetition must 
be spited in multiple parts to allow the notes to be set:

\begin{verbatim}
  Clock 128\,MHz 0\degr    & H         12{2C}             G \\ % without notes
  Clock 128\,MHz 0\degr    & H 2C N(A1) 8{2C} N(A5) 3{2C} G \\ % with    notes
\end{verbatim}

The lines are then drawn using the \texttt{foreach} loop of the PGF package.
The background layer is used to not overdraw the timing signals. The predefined 
style \qv{help lines} is used to get gray thin lines.
This code must be placed after the \verb+\extracode+ macro.

\begin{verbatim}
  \begin{pgfonlayer}{background}
    \foreach \n in {1,...,8}
      \draw [help lines] (A\n) -- (B\n);
  \end{pgfonlayer}
\end{verbatim}

This shows that the consistent naming of the nodes makes the drawing code very 
simple.

\subsection{Adding Table Rules}
To add \texttt{booktab} like table rules simply add the command 
\verb+\tablerules+ to the \texttt{extracode} section.

\section{Final Code and Result}

\begin{verbatim}
\def\degr{${}^\circ$}
\begin{tikztimingtable}
  Clock 128\,MHz 0\degr    & H 2C N(A1) 8{2C} N(A5) 3{2C} G\\
  Clock 128\,MHz 90\degr   & [C] 2{2C} N(A2) 8{2C} N(A6) 2{2C} C\\
  Clock 128\,MHz 180\degr  & C 2{2C} N(A3) 8{2C} N(A7) 2{2C} G\\
  Clock 128\,MHz 270\degr  & 3{2C} N(A4) 8{2C} N(A8) 2C C\\
  Coarse Pulse             & 3L 16H 6L \\
  Coarse Pulse - Delayed 1 & 4L N(B2) 16H N(B6) 5L \\
  Coarse Pulse - Delayed 2 & 5L N(B3) 16H N(B7) 4L \\
  Coarse Pulse - Delayed 3 & 6L 16H 3L \\
  \\
  Final Pulse Set          & 3L 16H N(B5) 6L \\
  Final Pulse $\overline{\mbox{Reset}}$ & 6L N(B4) 16H 3L \\
  Final Pulse              & 3L N(B1) 19H N(B8) 3L \\
\extracode
  \tablerules
  \begin{pgfonlayer}{background}
    \foreach \n in {1,...,8}
      \draw [help lines] (A\n) -- (B\n);
  \end{pgfonlayer}
\end{tikztimingtable}
\end{verbatim}
\fi % ifcomment

\def\degr{${}^\circ$}
\begin{tikztimingtable}
  PHASE A    			& H 2C N(A1) 8{2C} N(A5) 3{2C} G\\
  PHASE B   			& [C] 2{2C} N(A2) 8{2C} N(A6) 2{2C} C\\
  K Input Of Flipflop  	& C 2{2C} N(A3) 8{2C} N(A7) 2{2C} G\\
  DIR  					& 3{2C} N(A4) 8{2C} N(A8) 2C C\\
  STEP   				& 3L 16H 6L \\
  Coarse Pulse - Delayed 1 & 2L 0.5H 0.5L 0.5H 0.5L N(B2) 16H N(B6) 5L \\
  Coarse Pulse - Delayed 2 & 5L N(B3) 16H N(B7) 4L \\
  Coarse Pulse - Delayed 3 & 6L 16H 3L \\
  \\
  Final Pulse Set          & 3L 16H N(B5) 6L \\
  Final Pulse $\overline{\mbox{Reset}}$ & 6L N(B4) 16H 3L \\
                & 3L N(B1) 19H N(B8) 3L \\
\extracode
  \tablerules
  \begin{pgfonlayer}{background}
    \foreach \n in {1,...,8}
      \draw [help lines] (A\n) -- (B\n);
  \end{pgfonlayer}
\end{tikztimingtable}
%
\end{document}