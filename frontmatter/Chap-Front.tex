\begin{abstract}[english]%===================================================
In this report a design method for the swinging and balancing  of the underactuated robotic gymnast was researched, simulated and tested on a physical model. The electronic, mechanical and software designs are discussed to show how the physical model was constructed, controllers implemented and data acquired.
\end{abstract}


\begin{abstract}[afrikaans]%=================================================
In die projek word die swaaiende en balanseering beheerwette vir 'n robotiese gimnas genavors, ontwerp en getoets op 'n fisiese model. Die eletroniese, meganiese en sagteware ontwerpe word bespreek om ten einde te wys hoe die fisiese model, beheerders en so voort geimplementeer en getoets is.
\end{abstract}


\chapter{Acknowledgements}%==================================================

I would like to express my sincere gratitude to the following people
and organisations.\\

Dr. Japie Engelbrecht for supervising the project throughout the year. He provided critical feedback on progress, guided me in the correct direction and was a voice of reason throughout the project.\\

The Electrical and Electronic Department for allowing the use of their equipment and facilities. They allowed me to work with confidence ensuring the facilities and equipment are maintained with no interruption to my work.\\

Mr$.$ Croukamp, Mr$.$ van Eenden and Mr$.$ Lecholo for assisting me with the mechanical and electronic designs. The atmosphere in the manufacturing labotorium was always welcoming.\\

The group of final year student working together in the 4th floor labs. You all made the experience more worthwhile and encouraged me through  difficult times. 




\chapter{Dedications}%=======================================================
 \vfill
 \begin{Afr}
 \begin{center}\itshape
    Hierdie verslag word opgedra aan my ouers wat my ondersteun het tydens die 4 jaar om die graad van ingenieurswese te ontvang en die Here vir sy genade. 
 \end{center}
 \end{Afr}
 \vfill
 \clearpage

%============================================================================
\endinput
