\begin{titlingpage}


\newcolumntype{P}[1]{>{\centering\arraybackslash}p{#1}}
\chapter{Executive Summary}
\begin{table}[!h]
	\centering
	\begin{tabular}{|P{15cm}|}
		\hline
		\textbf{Executive Summary} \\
		\hline
		\hline
		\textbf{Project Title} \\
		\hline 
		The feedback control of a robotic gymnast \\
		\hline
		\textbf{Objectives} \\
		\hline 
		Design, implement and verify a feedback control system that is able to swing and balance a underactuated robotic gymnast. A physical system, consisting of mechanical and electronic hardware, must be designed and constructed, and the feedback controllers implemented in software. Verify the feedback control system in simulation using the physical system characteristics.\\
		\hline
		\textbf{Which aspects of the project are unique?} \\
		\hline
		 The electronic hardware that translates sensory signals into information that is used by the control system. The software that allows the testing of the system and acquisition of system information. The mechanical hardware that allows continuous rotation of the robotic gymnast without the loss of system information. \\
		 The implementation of the simulation of the robotic gymnast that visually presents the response of the system \\
		\hline
		\textbf{What are the (expected) findings?} \\
		\hline
		The simulation is an acceptable representation of the physical system and the control system is able to swing and balance in simulation. A robotic gymnast system that consist out of mechanical-, electronic hardware with software to test the control systems practically.\\
		\hline 
		\textbf{What value do the results have?} \\
		\hline
		Further development on the simulated model can continue to test other feedback control system to swing and balance the robotic gymnast. These control systems can be practically tested on the robotic gymnast system to determine how well they perform.\\
		\hline 
		\textbf{If more than one student is involved, what is each one's contribution?} \\
		\hline 
		Not Applicable \\
		\hline 
		\textbf{Which aspects of the project will carry on after completion?} \\
		\hline
		The robotic gymnast hardware will serve as a testbed for testing and demonstrating different swing up and balancing feedback control algorithms. \\
		\hline 
		\textbf{What are the expected advantages of continuation?} \\
		\hline 
		 Different feedback control algorithms to perform the swing up and balancing of the robotic gymnast can be evaluated and compared. \\
		\hline 
		\textbf{What arrangements have been made to expedite continuation?} \\
		\hline 
		The robotic gymnast mechanical hardware, electronics, and software will be given to the study leader for use in future projects. \\
		\hline
		
		
		
		
		
	\end{tabular}
\end{table}
\end{titlingpage}